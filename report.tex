\documentclass[a4paper,10pt]{report}
\usepackage[utf8]{inputenc}

% Title Page
\title{CANDLE syndrome}
\author{Juan Luis Melero and Helena Rodríguez}


\begin{document}
\maketitle
\tableofcontents

\begin{abstract}
\end{abstract}

\chapter{General Features}

\section{Denominations of the disease}

CANDLE disease is an acronym for \textit{Chronic Atypical Neutrophilic Dermatosis with Lipodystrophy and Elevated temperature}. This name was set by Torrelo et al. in 2010 \cite{Torrelo2010}.
However, this disease was previously described as \textit{Joint contractures, Muscle atrophy, Microcytic anemia and Panniculitis induced lipodystrophy (JMP)}, \textit{Japanese Autoinflammatory Syndrome with Lipodystrophy} and \textit{Nakajo-Nishimura Syndrome}.

\section{Frequency and genetics}

CANDLE syndrome us a rare disease, with a prevalence smaller than 1 affected person every million. Actually, only around 30 persons had been diagnosed with CANDLE syndrome in 2015, with no specific geographical distribution.\\

It can be caused by mutations in different genes, but the most common causative mutations are found in PSMB8, a subunit of the proteaseome. However, mutations in other subunits of the proteasome or a related protein that helps its ensembling can also cause the disease. These mutations are usually autosomic recessive, but we can also find them as heterozygous compounds or with autosomic dominant inheritance.

\section{Symptomes}


\end{document}          
