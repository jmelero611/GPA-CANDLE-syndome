\documentclass[a4paper,10pt]{report}
\usepackage[utf8]{inputenc}

% Title Page
\title{CANDLE syndrome}
\author{Juan Luis Melero and Helena Rodríguez}


\begin{document}
\maketitle
\tableofcontents

\begin{abstract}
\end{abstract}

\chapter{General Features}

\section{Denominations of the disease}

CANDLE disease is an acronym for \textit{Chronic Atypical Neutrophilic Dermatosis with Lipodystrophy and Elevated temperature}. This name was set by Torrelo et al. in 2010 \cite{Torrelo2010}.
However, this disease was previously described as \textit{Joint contractures, Muscle atrophy, Microcytic anemia and Panniculitis induced lipodystrophy (JMP)}, \textit{Japanese Autoinflammatory Syndrome with Lipodystrophy} and \textit{Nakajo-Nishimura Syndrome}.

\section{Frequency and genetics}

CANDLE syndrome us a rare disease, with a prevalence smaller than 1 affected person every million. Actually, only around 30 persons had been diagnosed with CANDLE syndrome in 2015, with no specific geographical distribution.\par
It can be caused by mutations in different genes, but the most common causative mutations are found in PSMB8, a subunit of the proteaseome. However, mutations in other subunits of the proteasome or a related protein that helps its ensembling can also cause the disease. These mutations are usually autosomic recessive, but we can also find them as heterozygous compounds or with autosomic dominant inheritance.

\section{Symptomes}
\subsection{Common symptomes}

The first symptomes of the disease appear within the first months of life, and consinst on recurrent and almost daily fevers and anular erythematous skin lesions (fig1), which are less prominent after puberty. Another skin manifestation, common specially during childhood, is persistent eyelid swelling, which fades aftes days or weeks of each appearance.\par
CANDLE patients also have a mild to moderate growth delay, with low weight and height, due to chronic inflammation and lipodistrophy; usually without mental retardation. This lipodistrophy is appears when patients are around two years old, and progresses over time, from the face to the trunk and upper limbs, while lower limbs are usually less affected. It is thought to be due to the chronic inflammation, and the cause of another symptome, hepatomegaly. \par
Finally, the last common symptome is arthralgia, or joint pain, which can cause contractures in adults and some disability.

\subsection{Less common symptomes}

Other symptomes may appear depending on the progress of the disease and the organs affected by accute inflamatory events that might take place during the patients life. Some examples are:
\begin{itemsize}
  \item Splenomegaly and limphadenopathy, caused by the persistent autoinflammatory activity.
  \item Chronic condritis of ears and nose, probably caused due to beign exposed to cold wheather.
  \item Different types of meningitis and central nervous system inflammaion.
  \item Persistent generalised inflammation with acute inflammation episodes that may affect any organ.
  \item Acute myositis attacks, probably with some chronic muscle inflammation.
\end{itemsize}
These symptomes may cause some degree of disability or even death.

\section{Diagnosis}

Nowadays, patients are diagnosed when the first symptomes (fevers, skin lessions and lipodistrophy) appear. The first thing to do is a lab analysis to analyse the skin lessions, and when de doctor suspects of CANDLE, sequence PSMB8.\par
If the result is positive the patient is diagnosed with CANDLE. If it is negative the next step is to look at the other genes that may cause the disease. A positive result will also result in the diagnosis of CANDLE, will a negative result will require more analysis, to check for other disease or other causes of the disease.\par
Finally, we can also sequence relatives, to find out the inheritance.

\chapter{Genetic variants of the disease}

\section{Genes and mutation}
\subsection{PSMB8}

PSMB8 is the gene where most CANDLE mutations are found. It is located in the short arm of chromosome 6 (6p21.32) and encodes for the subunit β5i of the proteasome, which has chemotrypsin-like activity and is crutial to the immunoproteasome's function.\par
Several mutations in this gene can cause CANDLE syndrome. Most of them cause an amino-acid change, and the most common one is T75M, which causes select impairment of the chemotrypic activity, like A92T, K105Q and M117V.\par
K105Q is also related to poor incorporation and maduration of the proteasome, and inhability to completely trim the β5i pro-peptide. T75M and G201V also cause low proteasome assembly.\par
Finally, C135X causes the truncation and non-expression of the protein, which impairs the immunoproteasome assembly and reduces all three proteasome aciticities (trypsin-like, caspase-like and chemotrypsin-like).\par
For all cases, these are the effects when the mutation is found in homozygosis.

\subsection{PSMA3}

Another gene which may cause CANDLE syndrome is PSMA3, located in the long arm of chromosoem 14 (14q23.1) and encodes for the subunit α7 of the proteasome.\par
In this case there are two described mutations. The first one, p.R233del (c.696_698delAAG), is the deletion of an arginin which affects the folding of the subunit and its incorporation to the mature proteasome, reducing the proteasome content.\par
The other one affects the exon 5's splicing site, c.404+2T>C, which is skipped. This causes and unstable transcript and affects proteaosme assembling.\par
Again, these are autosomic recessive mutations.

\subsection{PSMB4}

PSMB4 can also cause the disease. It is located in the long arm of chromosome 1 (1q21) and encodes for the subunit β7 of the proteasome, important for proteasome assembly and stabilization.\par
The -9G>A mutation causes a protein with lower expression that is less incorporated into the proteasome than the WT and may impair the cleavage of the subunit β5i.\par
The other mutations affect the C-terminal extension, essential for the proteasome assembly. The p.D212-V214 deletion affects the N-terminus, which helps C-term stabilization. The c.44insG (p.16Sfs*45) mutation is an insertion that causes a frameshift mutation that leads to the non-expression if the allele; and the miss-sense mutation p.Y222X causes the deletion of the C-terminus. In all cases, this results in poorly incorporation of the subunit into the proteasome.

\end{document}          
